\section*{\emph{Checksums} e \emph{Snapshots} do BTRFS}

O sistema de arquivos presente na \emph{kernel Linux} chamado BTRFS possui nativamente
suporte para \emph{Checksums}. Esses permitem que o \emph{filesystem} ativamente corrija
blocos de dados defeituosos e corrigi-los automaticamente.

Caso algo saia do esperado é possível usar a ferramenta de \emph{snapshots} que esse \emph{filesystem}
possui para voltar o backup do sistema e recuperar os dados perdidos. Esses \emph{snapshots} são
feitos em nível de sistema, o que significa que caso algum arquivo até em nível de \emph{kernel} tenha sido
modificado, poderá ser recuperado.

\section*{\emph{fsck} do sistema de arquivos ext4}

O comando \emph{fsck} é conhecido por fazer diagnóstico de \emph{badblocks} e recuperá-los 
caso seja encontrado que estão corrompidos. Diferente do sistema de arquivos acima, o comando
precisa ser executado manualmente pelo administrador do sistema, o que pode acarretar em um grande acumulo
de erros no sistema de arquivos até o ponto de ser irrecuperável.

\section*{\emph{Golang Diagnostics}}

Um sistema de diagnóstico criado pela \emph{Google} dentro de sua língua de programação \emph{Go} chamado
\href{https://go.dev/doc/diagnostics}{\emph{Diagnostics Go Dev}}. Consegue diagnosticar o uso da CPU, 
o \emph{heap} da memória, a criação de \emph{threads} do sistema operacional, a criação de \emph{green threads}
do \emph{go} chamada \emph{goroutines} e seus \emph{locks} e \emph{mutexes}.

Esse conjunto de ferramentes são capazes de diagnosticar com precisão um sistema. Com o uso deles é possível
evitar problemas no sistema em tempo real, tanto quanto gerar \emph{logs} detalhados sobre todos os eventos 
que estão acontecendo nesse sistema.


\section*{Caso queira referênciar este arquivo}

Este documento está disponível sob a licença \href{https://choosealicense.com/licenses/mit/}{MIT} no repositório
\href{https://github.com/Trees-Over-The-Lake/Computacao-Distribuida/blob/main/Tarefa%205/main.pdf}{Computação Distribuída - Github}