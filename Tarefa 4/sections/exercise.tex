O problema dos dois generais é um problema muito conhecido. Esse problema consiste em dois generais
que querem atacar um mesmo forte. Cada um dos generais está de um lado do forte sem conseguir se comunicar
facilmente com o outro, sendo necessário enviar um mensageiro. Eles sabem que se atacarem separadamente irão perder. Com isso, eles precisam
se organizar para atacar simultaneamente. Sabe-se, também, que enviar um mensageiro não garante que ele chegue do outro lado, pois pode ser
atacado durante a travessia. Para garantir então que a mensagem foi um sucesso, pode-se adotar que um segundo mensageiro seja mandado do general
que acabou de receber a mensagem para o primeiro.

Esse problema não possui solução, no momento em que for enviado o segundo mensageiro para confirmar o recebimento da mensagem,
pode acontecer do segundo mensageiro ser atacado na travessia e não avisar para o primeiro general. Com isso, o primeiro general não teria certeza
se sua mensagem chegou corretamente e mais uma vez o ataque não funcionaria. Enviar mais mensageiros não soluciona o problema, pois quando o segundo
mensageiro chegar no primeiro general, então seria necessário um terceiro mensageiro para confirmar sua chegado
para o segundo general e assim por diante.

Protocolos UDP não sofrem com esse problema, pois quando o pacote é enviado, não se pede confirmações. Então, se perdido não haverá
confirmações enviadas por nenhuma das partes. Entretando, ao mudar o foco para o protocolo TCP, o problema aparece. Exigindo confirmações
para abrir e fechar conexões pode acarretar no problema de confirmações infinitas. A solução adotado para o TCP é considerar que a conexão
foi um sucesso, mesmo sem ter recebido a confirmação. Para considerar que falhou a conexão é colocado um tempo de \emph{timeout} para então
reenviar uma tentativa de conexão.


\vspace{1cm}
\noindent Referências:

\noindent Vídeo explicativo para ilustrar: \href{https://www.youtube.com/watch?v=IP-rGJKSZ3s}{The Two General's Problem - Tom Scott}

\noindent Discução no Stackoverflow: \href{https://stackoverflow.com/questions/36352236/two-general-agreement-and-tcp-handshake}{Two general agreement and TCP handshake}