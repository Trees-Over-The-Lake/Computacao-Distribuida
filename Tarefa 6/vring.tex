\documentclass{article} 
\usepackage{tikz}
\usepackage[portuguese]{babel}
\usepackage{graphicx}

\newcommand{\ok}{\includegraphics[scale=0.25]{emojis/tick.png}}
\newcommand{\fail}{\includegraphics[scale=0.25]{emojis/cross.png}}

\usetikzlibrary{automata,positioning,shapes.multipart}

\title{Algoritmo de VRing com erro no node 3}
\author{Lucas Santiago de Oliveira}
\date{\today}

\begin{document} 
    \maketitle

    Aqui estão os 7 processos. Nenhum deles tem conhecimento do funcionamento dos outros.
    \vspace{1cm}

    \begin{tikzpicture}[node distance={20mm}, thick, main/.style = {draw, circle}] 
        \node[main] (1) {$x_1$};
        \node[above left of=1, rectangle] (desc1) {
            \begin{tabular}{|c|c|}
                \hline
                $x_2$ &  \\
                \hline
                $x_3$ &  \\
                \hline
                $x_4$ &  \\
                \hline
                $x_5$ &  \\
                \hline
                $x_6$ &  \\
                \hline
                $x_7$ &  \\
                \hline
            \end{tabular}
        };
        \node[main] (2) [above right of=1] {$x_2$};
        \node[above of=2, rectangle] (desc2) {
            \begin{tabular}{|c|c|}
                \hline
                $x_1$ &  \\
                \hline
                $x_3$ &  \\
                \hline
                $x_4$ &  \\
                \hline
                $x_5$ &  \\
                \hline
                $x_6$ &  \\
                \hline
                $x_7$ &  \\
                \hline
            \end{tabular}
        };
        \node[main] (3) [right of=2] {$x_3$}; 
        \node[above of=3, rectangle] (desc3) {
            \begin{tabular}{|c|c|}
                \hline
                $x_1$ &  \\
                \hline
                $x_2$ &  \\
                \hline
                $x_4$ &  \\
                \hline
                $x_5$ &  \\
                \hline
                $x_6$ &  \\
                \hline
                $x_7$ &  \\
                \hline
            \end{tabular}
        };
        \node[main] (4) [below right of=3] {$x_4$};
        \node[right of=4, rectangle] (desc4) {
            \begin{tabular}{|c|c|}
                \hline
                $x_1$ &  \\
                \hline
                $x_2$ &  \\
                \hline
                $x_3$ &  \\
                \hline
                $x_5$ &  \\
                \hline
                $x_6$ &  \\
                \hline
                $x_7$ &  \\
                \hline
            \end{tabular}
        };
        \node[main] (5) [below left of=4] {$x_5$}; 
        \node[below of=5, rectangle] (desc5) {
            \begin{tabular}{|c|c|}
                \hline
                $x_1$ &  \\
                \hline
                $x_2$ &  \\
                \hline
                $x_3$ &  \\
                \hline
                $x_4$ &  \\
                \hline
                $x_6$ &  \\
                \hline
                $x_7$ &  \\
                \hline
            \end{tabular}
        };
        \node[main] (6) [left of=5] {$x_6$};
        \node[below of=6, rectangle] (desc6) {
            \begin{tabular}{|c|c|}
                \hline
                $x_1$ &  \\
                \hline
                $x_2$ &  \\
                \hline
                $x_3$ &  \\
                \hline
                $x_4$ &  \\
                \hline
                $x_5$ &  \\
                \hline
                $x_7$ &  \\
                \hline
            \end{tabular}
        };
        \node[main] (7) [left of=6] {$x_7$};
        \node[below of=7, rectangle] (desc7) {
            \begin{tabular}{|c|c|}
                \hline
                $x_1$ &  \\
                \hline
                $x_2$ &  \\
                \hline
                $x_3$ &  \\
                \hline
                $x_4$ &  \\
                \hline
                $x_5$ &  \\
                \hline
                $x_6$ &  \\
                \hline
            \end{tabular}
        };
    \end{tikzpicture}

    \vspace{1cm}
    Aqui o processo 1 verifica o processo 2.
    \vspace{1cm}

    \begin{tikzpicture}[node distance={20mm}, thick, main/.style = {draw, circle}] 
        \node[main] (1) {$x_1$};
        \node[above left of=1, rectangle] (desc1) {
            \begin{tabular}{|c|c|}
                \hline
                $x_2$ & \ok \\
                \hline
                $x_3$ &  \\
                \hline
                $x_4$ &  \\
                \hline
                $x_5$ &  \\
                \hline
                $x_6$ &  \\
                \hline
                $x_7$ &  \\
                \hline
            \end{tabular}
        };
        \node[main] (2) [above right of=1] {$x_2$};
        \node[above of=2, rectangle] (desc2) {
            \begin{tabular}{|c|c|}
                \hline
                $x_1$ &  \\
                \hline
                $x_3$ &  \\
                \hline
                $x_4$ &  \\
                \hline
                $x_5$ &  \\
                \hline
                $x_6$ &  \\
                \hline
                $x_7$ &  \\
                \hline
            \end{tabular}
        };
        \node[main] (3) [right of=2] {$x_3$}; 
        \node[above of=3, rectangle] (desc3) {
            \begin{tabular}{|c|c|}
                \hline
                $x_1$ &  \\
                \hline
                $x_2$ &  \\
                \hline
                $x_4$ &  \\
                \hline
                $x_5$ &  \\
                \hline
                $x_6$ &  \\
                \hline
                $x_7$ &  \\
                \hline
            \end{tabular}
        };
        \node[main] (4) [below right of=3] {$x_4$};
        \node[right of=4, rectangle] (desc4) {
            \begin{tabular}{|c|c|}
                \hline
                $x_1$ &  \\
                \hline
                $x_2$ &  \\
                \hline
                $x_3$ &  \\
                \hline
                $x_5$ &  \\
                \hline
                $x_6$ &  \\
                \hline
                $x_7$ &  \\
                \hline
            \end{tabular}
        };
        \node[main] (5) [below left of=4] {$x_5$}; 
        \node[below of=5, rectangle] (desc5) {
            \begin{tabular}{|c|c|}
                \hline
                $x_1$ &  \\
                \hline
                $x_2$ &  \\
                \hline
                $x_3$ &  \\
                \hline
                $x_4$ &  \\
                \hline
                $x_6$ &  \\
                \hline
                $x_7$ &  \\
                \hline
            \end{tabular}
        };
        \node[main] (6) [left of=5] {$x_6$};
        \node[below of=6, rectangle] (desc6) {
            \begin{tabular}{|c|c|}
                \hline
                $x_1$ &  \\
                \hline
                $x_2$ &  \\
                \hline
                $x_3$ &  \\
                \hline
                $x_4$ &  \\
                \hline
                $x_5$ &  \\
                \hline
                $x_7$ &  \\
                \hline
            \end{tabular}
        };
        \node[main] (7) [left of=6] {$x_7$};
        \node[below of=7, rectangle] (desc7) {
            \begin{tabular}{|c|c|}
                \hline
                $x_1$ &  \\
                \hline
                $x_2$ &  \\
                \hline
                $x_3$ &  \\
                \hline
                $x_4$ &  \\
                \hline
                $x_5$ &  \\
                \hline
                $x_6$ &  \\
                \hline
            \end{tabular}
            };
            \draw[->] (1) -- (2);
    \end{tikzpicture}

    \vspace{1cm}
    Aqui o processo 2 verifica o processo 3.
    \vspace{1cm}

    \begin{tikzpicture}[node distance={20mm}, thick, main/.style = {draw, circle}] 
        \node[main] (1) {$x_1$};
        \node[above left of=1, rectangle] (desc1) {
            \begin{tabular}{|c|c|}
                \hline
                $x_2$ & \ok \\
                \hline
                $x_3$ &  \\
                \hline
                $x_4$ &  \\
                \hline
                $x_5$ &  \\
                \hline
                $x_6$ &  \\
                \hline
                $x_7$ &  \\
                \hline
            \end{tabular}
        };
        \node[main] (2) [above right of=1] {$x_2$};
        \node[above of=2, rectangle] (desc2) {
            \begin{tabular}{|c|c|}
                \hline
                $x_1$ &  \\
                \hline
                $x_3$ & \fail \\
                \hline
                $x_4$ & \\
                \hline
                $x_5$ &  \\
                \hline
                $x_6$ &  \\
                \hline
                $x_7$ &  \\
                \hline
            \end{tabular}
        };
        \node[main] (3) [right of=2] {$x_3$}; 
        \node[above of=3, rectangle] (desc3) {
            \begin{tabular}{|c|c|}
                \hline
                $x_1$ &  \\
                \hline
                $x_2$ &  \\
                \hline
                $x_4$ &  \\
                \hline
                $x_5$ &  \\
                \hline
                $x_6$ &  \\
                \hline
                $x_7$ &  \\
                \hline
            \end{tabular}
        };
        \node[main] (4) [below right of=3] {$x_4$};
        \node[right of=4, rectangle] (desc4) {
            \begin{tabular}{|c|c|}
                \hline
                $x_1$ &  \\
                \hline
                $x_2$ &  \\
                \hline
                $x_3$ &  \\
                \hline
                $x_5$ &  \\
                \hline
                $x_6$ &  \\
                \hline
                $x_7$ &  \\
                \hline
            \end{tabular}
        };
        \node[main] (5) [below left of=4] {$x_5$}; 
        \node[below of=5, rectangle] (desc5) {
            \begin{tabular}{|c|c|}
                \hline
                $x_1$ &  \\
                \hline
                $x_2$ &  \\
                \hline
                $x_3$ &  \\
                \hline
                $x_4$ &  \\
                \hline
                $x_6$ &  \\
                \hline
                $x_7$ &  \\
                \hline
            \end{tabular}
        };
        \node[main] (6) [left of=5] {$x_6$};
        \node[below of=6, rectangle] (desc6) {
            \begin{tabular}{|c|c|}
                \hline
                $x_1$ &  \\
                \hline
                $x_2$ &  \\
                \hline
                $x_3$ &  \\
                \hline
                $x_4$ &  \\
                \hline
                $x_5$ &  \\
                \hline
                $x_7$ &  \\
                \hline
            \end{tabular}
        };
        \node[main] (7) [left of=6] {$x_7$};
        \node[below of=7, rectangle] (desc7) {
            \begin{tabular}{|c|c|}
                \hline
                $x_1$ &  \\
                \hline
                $x_2$ &  \\
                \hline
                $x_3$ &  \\
                \hline
                $x_4$ &  \\
                \hline
                $x_5$ &  \\
                \hline
                $x_6$ &  \\
                \hline
            \end{tabular}
            };
            \draw[->] (1) -- (2);
            \draw[->] (2) -- (3);
    \end{tikzpicture}

    \vspace{1cm}
    Aqui o processo 2 verifica o processo 4, pois o \emph{node} 3 não está respondedo corretamente.
    Verfica e aprova o funcionamento do processo 4.
    \vspace{1cm}

    \begin{tikzpicture}[node distance={20mm}, thick, main/.style = {draw, circle}] 
        \node[main] (1) {$x_1$};
        \node[above left of=1, rectangle] (desc1) {
            \begin{tabular}{|c|c|}
                \hline
                $x_2$ & \ok \\
                \hline
                $x_3$ &  \\
                \hline
                $x_4$ &  \\
                \hline
                $x_5$ &  \\
                \hline
                $x_6$ &  \\
                \hline
                $x_7$ &  \\
                \hline
            \end{tabular}
        };
        \node[main] (2) [above right of=1] {$x_2$};
        \node[above of=2, rectangle] (desc2) {
            \begin{tabular}{|c|c|}
                \hline
                $x_1$ &  \\
                \hline
                $x_3$ & \fail \\
                \hline
                $x_4$ & \ok \\
                \hline
                $x_5$ &  \\
                \hline
                $x_6$ &  \\
                \hline
                $x_7$ &  \\
                \hline
            \end{tabular}
        };
        \node[main] (3) [right of=2] {$x_3$}; 
        \node[above of=3, rectangle] (desc3) {
            \begin{tabular}{|c|c|}
                \hline
                $x_1$ &  \\
                \hline
                $x_2$ &  \\
                \hline
                $x_4$ &  \\
                \hline
                $x_5$ &  \\
                \hline
                $x_6$ &  \\
                \hline
                $x_7$ &  \\
                \hline
            \end{tabular}
        };
        \node[main] (4) [below right of=3] {$x_4$};
        \node[right of=4, rectangle] (desc4) {
            \begin{tabular}{|c|c|}
                \hline
                $x_1$ &  \\
                \hline
                $x_2$ &  \\
                \hline
                $x_3$ &  \\
                \hline
                $x_5$ &  \\
                \hline
                $x_6$ &  \\
                \hline
                $x_7$ &  \\
                \hline
            \end{tabular}
        };
        \node[main] (5) [below left of=4] {$x_5$}; 
        \node[below of=5, rectangle] (desc5) {
            \begin{tabular}{|c|c|}
                \hline
                $x_1$ &  \\
                \hline
                $x_2$ &  \\
                \hline
                $x_3$ &  \\
                \hline
                $x_4$ &  \\
                \hline
                $x_6$ &  \\
                \hline
                $x_7$ &  \\
                \hline
            \end{tabular}
        };
        \node[main] (6) [left of=5] {$x_6$};
        \node[below of=6, rectangle] (desc6) {
            \begin{tabular}{|c|c|}
                \hline
                $x_1$ &  \\
                \hline
                $x_2$ &  \\
                \hline
                $x_3$ &  \\
                \hline
                $x_4$ &  \\
                \hline
                $x_5$ &  \\
                \hline
                $x_7$ &  \\
                \hline
            \end{tabular}
        };
        \node[main] (7) [left of=6] {$x_7$};
        \node[below of=7, rectangle] (desc7) {
            \begin{tabular}{|c|c|}
                \hline
                $x_1$ &  \\
                \hline
                $x_2$ &  \\
                \hline
                $x_3$ &  \\
                \hline
                $x_4$ &  \\
                \hline
                $x_5$ &  \\
                \hline
                $x_6$ &  \\
                \hline
            \end{tabular}
            };
            \draw[->] (1) -- (2);
            \draw[->] (2) -- (3);
            \draw[->] (2) -- (4);
    \end{tikzpicture}

    \vspace{1cm}
    Aqui o processo 4 verifica o processo 5.
    \vspace{1cm}

    \begin{tikzpicture}[node distance={20mm}, thick, main/.style = {draw, circle}] 
        \node[main] (1) {$x_1$};
        \node[above left of=1, rectangle] (desc1) {
            \begin{tabular}{|c|c|}
                \hline
                $x_2$ & \ok \\
                \hline
                $x_3$ &  \\
                \hline
                $x_4$ &  \\
                \hline
                $x_5$ &  \\
                \hline
                $x_6$ &  \\
                \hline
                $x_7$ &  \\
                \hline
            \end{tabular}
        };
        \node[main] (2) [above right of=1] {$x_2$};
        \node[above of=2, rectangle] (desc2) {
            \begin{tabular}{|c|c|}
                \hline
                $x_1$ &  \\
                \hline
                $x_3$ & \fail \\
                \hline
                $x_4$ & \ok \\
                \hline
                $x_5$ &  \\
                \hline
                $x_6$ &  \\
                \hline
                $x_7$ &  \\
                \hline
            \end{tabular}
        };
        \node[main] (3) [right of=2] {$x_3$}; 
        \node[above of=3, rectangle] (desc3) {
            \begin{tabular}{|c|c|}
                \hline
                $x_1$ &  \\
                \hline
                $x_2$ &  \\
                \hline
                $x_4$ &  \\
                \hline
                $x_5$ &  \\
                \hline
                $x_6$ &  \\
                \hline
                $x_7$ &  \\
                \hline
            \end{tabular}
        };
        \node[main] (4) [below right of=3] {$x_4$};
        \node[right of=4, rectangle] (desc4) {
            \begin{tabular}{|c|c|}
                \hline
                $x_1$ &  \\
                \hline
                $x_2$ &  \\
                \hline
                $x_3$ &  \\
                \hline
                $x_5$ & \ok \\
                \hline
                $x_6$ &  \\
                \hline
                $x_7$ &  \\
                \hline
            \end{tabular}
        };
        \node[main] (5) [below left of=4] {$x_5$}; 
        \node[below of=5, rectangle] (desc5) {
            \begin{tabular}{|c|c|}
                \hline
                $x_1$ &  \\
                \hline
                $x_2$ &  \\
                \hline
                $x_3$ &  \\
                \hline
                $x_4$ &  \\
                \hline
                $x_6$ &  \\
                \hline
                $x_7$ &  \\
                \hline
            \end{tabular}
        };
        \node[main] (6) [left of=5] {$x_6$};
        \node[below of=6, rectangle] (desc6) {
            \begin{tabular}{|c|c|}
                \hline
                $x_1$ &  \\
                \hline
                $x_2$ &  \\
                \hline
                $x_3$ &  \\
                \hline
                $x_4$ &  \\
                \hline
                $x_5$ &  \\
                \hline
                $x_7$ &  \\
                \hline
            \end{tabular}
        };
        \node[main] (7) [left of=6] {$x_7$};
        \node[below of=7, rectangle] (desc7) {
            \begin{tabular}{|c|c|}
                \hline
                $x_1$ &  \\
                \hline
                $x_2$ &  \\
                \hline
                $x_3$ &  \\
                \hline
                $x_4$ &  \\
                \hline
                $x_5$ &  \\
                \hline
                $x_6$ &  \\
                \hline
            \end{tabular}
            };
            \draw[->] (1) -- (2);
            \draw[->] (2) -- (3);
            \draw[->] (2) -- (4);
            \draw[->] (4) -- (5);
    \end{tikzpicture}

    \vspace{1cm}
    Aqui o processo 5 verifica o processo 6.
    \vspace{1cm}

    \begin{tikzpicture}[node distance={20mm}, thick, main/.style = {draw, circle}] 
        \node[main] (1) {$x_1$};
        \node[above left of=1, rectangle] (desc1) {
            \begin{tabular}{|c|c|}
                \hline
                $x_2$ & \ok \\
                \hline
                $x_3$ &  \\
                \hline
                $x_4$ &  \\
                \hline
                $x_5$ &  \\
                \hline
                $x_6$ &  \\
                \hline
                $x_7$ &  \\
                \hline
            \end{tabular}
        };
        \node[main] (2) [above right of=1] {$x_2$};
        \node[above of=2, rectangle] (desc2) {
            \begin{tabular}{|c|c|}
                \hline
                $x_1$ &  \\
                \hline
                $x_3$ & \fail \\
                \hline
                $x_4$ & \ok \\
                \hline
                $x_5$ &  \\
                \hline
                $x_6$ &  \\
                \hline
                $x_7$ &  \\
                \hline
            \end{tabular}
        };
        \node[main] (3) [right of=2] {$x_3$}; 
        \node[above of=3, rectangle] (desc3) {
            \begin{tabular}{|c|c|}
                \hline
                $x_1$ &  \\
                \hline
                $x_2$ &  \\
                \hline
                $x_4$ &  \\
                \hline
                $x_5$ &  \\
                \hline
                $x_6$ &  \\
                \hline
                $x_7$ &  \\
                \hline
            \end{tabular}
        };
        \node[main] (4) [below right of=3] {$x_4$};
        \node[right of=4, rectangle] (desc4) {
            \begin{tabular}{|c|c|}
                \hline
                $x_1$ &  \\
                \hline
                $x_2$ &  \\
                \hline
                $x_3$ &  \\
                \hline
                $x_5$ & \ok \\
                \hline
                $x_6$ &  \\
                \hline
                $x_7$ &  \\
                \hline
            \end{tabular}
        };
        \node[main] (5) [below left of=4] {$x_5$}; 
        \node[below of=5, rectangle] (desc5) {
            \begin{tabular}{|c|c|}
                \hline
                $x_1$ &  \\
                \hline
                $x_2$ &  \\
                \hline
                $x_3$ &  \\
                \hline
                $x_4$ &  \\
                \hline
                $x_6$ & \ok \\
                \hline
                $x_7$ &  \\
                \hline
            \end{tabular}
        };
        \node[main] (6) [left of=5] {$x_6$};
        \node[below of=6, rectangle] (desc6) {
            \begin{tabular}{|c|c|}
                \hline
                $x_1$ &  \\
                \hline
                $x_2$ &  \\
                \hline
                $x_3$ &  \\
                \hline
                $x_4$ &  \\
                \hline
                $x_5$ &  \\
                \hline
                $x_7$ &  \\
                \hline
            \end{tabular}
        };
        \node[main] (7) [left of=6] {$x_7$};
        \node[below of=7, rectangle] (desc7) {
            \begin{tabular}{|c|c|}
                \hline
                $x_1$ &  \\
                \hline
                $x_2$ &  \\
                \hline
                $x_3$ &  \\
                \hline
                $x_4$ &  \\
                \hline
                $x_5$ &  \\
                \hline
                $x_6$ &  \\
                \hline
            \end{tabular}
            };
            \draw[->] (1) -- (2);
            \draw[->] (2) -- (3);
            \draw[->] (2) -- (4);
            \draw[->] (4) -- (5);
            \draw[->] (5) -- (6);
    \end{tikzpicture}

    \vspace{1cm}
    Aqui o processo 6 verifica o processo 7.
    \vspace{1cm}

    \begin{tikzpicture}[node distance={20mm}, thick, main/.style = {draw, circle}] 
        \node[main] (1) {$x_1$};
        \node[above left of=1, rectangle] (desc1) {
            \begin{tabular}{|c|c|}
                \hline
                $x_2$ & \ok \\
                \hline
                $x_3$ &  \\
                \hline
                $x_4$ &  \\
                \hline
                $x_5$ &  \\
                \hline
                $x_6$ &  \\
                \hline
                $x_7$ &  \\
                \hline
            \end{tabular}
        };
        \node[main] (2) [above right of=1] {$x_2$};
        \node[above of=2, rectangle] (desc2) {
            \begin{tabular}{|c|c|}
                \hline
                $x_1$ &  \\
                \hline
                $x_3$ & \fail \\
                \hline
                $x_4$ & \ok \\
                \hline
                $x_5$ &  \\
                \hline
                $x_6$ &  \\
                \hline
                $x_7$ &  \\
                \hline
            \end{tabular}
        };
        \node[main] (3) [right of=2] {$x_3$}; 
        \node[above of=3, rectangle] (desc3) {
            \begin{tabular}{|c|c|}
                \hline
                $x_1$ &  \\
                \hline
                $x_2$ &  \\
                \hline
                $x_4$ &  \\
                \hline
                $x_5$ &  \\
                \hline
                $x_6$ &  \\
                \hline
                $x_7$ &  \\
                \hline
            \end{tabular}
        };
        \node[main] (4) [below right of=3] {$x_4$};
        \node[right of=4, rectangle] (desc4) {
            \begin{tabular}{|c|c|}
                \hline
                $x_1$ &  \\
                \hline
                $x_2$ &  \\
                \hline
                $x_3$ &  \\
                \hline
                $x_5$ & \ok \\
                \hline
                $x_6$ &  \\
                \hline
                $x_7$ &  \\
                \hline
            \end{tabular}
        };
        \node[main] (5) [below left of=4] {$x_5$}; 
        \node[below of=5, rectangle] (desc5) {
            \begin{tabular}{|c|c|}
                \hline
                $x_1$ &  \\
                \hline
                $x_2$ &  \\
                \hline
                $x_3$ &  \\
                \hline
                $x_4$ &  \\
                \hline
                $x_6$ & \ok \\
                \hline
                $x_7$ &  \\
                \hline
            \end{tabular}
        };
        \node[main] (6) [left of=5] {$x_6$};
        \node[below of=6, rectangle] (desc6) {
            \begin{tabular}{|c|c|}
                \hline
                $x_1$ &  \\
                \hline
                $x_2$ &  \\
                \hline
                $x_3$ &  \\
                \hline
                $x_4$ &  \\
                \hline
                $x_5$ &  \\
                \hline
                $x_7$ & \ok \\
                \hline
            \end{tabular}
        };
        \node[main] (7) [left of=6] {$x_7$};
        \node[below of=7, rectangle] (desc7) {
            \begin{tabular}{|c|c|}
                \hline
                $x_1$ &  \\
                \hline
                $x_2$ &  \\
                \hline
                $x_3$ &  \\
                \hline
                $x_4$ &  \\
                \hline
                $x_5$ &  \\
                \hline
                $x_6$ &  \\
                \hline
            \end{tabular}
            };
            \draw[->] (1) -- (2);
            \draw[->] (2) -- (3);
            \draw[->] (2) -- (4);
            \draw[->] (4) -- (5);
            \draw[->] (5) -- (6);
            \draw[->] (6) -- (7);
    \end{tikzpicture}

    \vspace{1cm}
    Aqui o processo 7 verifica o processo 1.
    \vspace{1cm}

    \begin{tikzpicture}[node distance={20mm}, thick, main/.style = {draw, circle}] 
        \node[main] (1) {$x_1$};
        \node[above left of=1, rectangle] (desc1) {
            \begin{tabular}{|c|c|}
                \hline
                $x_2$ & \ok \\
                \hline
                $x_3$ &  \\
                \hline
                $x_4$ &  \\
                \hline
                $x_5$ &  \\
                \hline
                $x_6$ &  \\
                \hline
                $x_7$ &  \\
                \hline
            \end{tabular}
        };
        \node[main] (2) [above right of=1] {$x_2$};
        \node[above of=2, rectangle] (desc2) {
            \begin{tabular}{|c|c|}
                \hline
                $x_1$ &  \\
                \hline
                $x_3$ & \fail \\
                \hline
                $x_4$ & \ok \\
                \hline
                $x_5$ &  \\
                \hline
                $x_6$ &  \\
                \hline
                $x_7$ &  \\
                \hline
            \end{tabular}
        };
        \node[main] (3) [right of=2] {$x_3$}; 
        \node[above of=3, rectangle] (desc3) {
            \begin{tabular}{|c|c|}
                \hline
                $x_1$ &  \\
                \hline
                $x_2$ &  \\
                \hline
                $x_4$ &  \\
                \hline
                $x_5$ &  \\
                \hline
                $x_6$ &  \\
                \hline
                $x_7$ &  \\
                \hline
            \end{tabular}
        };
        \node[main] (4) [below right of=3] {$x_4$};
        \node[right of=4, rectangle] (desc4) {
            \begin{tabular}{|c|c|}
                \hline
                $x_1$ &  \\
                \hline
                $x_2$ &  \\
                \hline
                $x_3$ &  \\
                \hline
                $x_5$ & \ok \\
                \hline
                $x_6$ &  \\
                \hline
                $x_7$ &  \\
                \hline
            \end{tabular}
        };
        \node[main] (5) [below left of=4] {$x_5$}; 
        \node[below of=5, rectangle] (desc5) {
            \begin{tabular}{|c|c|}
                \hline
                $x_1$ &  \\
                \hline
                $x_2$ &  \\
                \hline
                $x_3$ &  \\
                \hline
                $x_4$ &  \\
                \hline
                $x_6$ & \ok \\
                \hline
                $x_7$ &  \\
                \hline
            \end{tabular}
        };
        \node[main] (6) [left of=5] {$x_6$};
        \node[below of=6, rectangle] (desc6) {
            \begin{tabular}{|c|c|}
                \hline
                $x_1$ &  \\
                \hline
                $x_2$ &  \\
                \hline
                $x_3$ &  \\
                \hline
                $x_4$ &  \\
                \hline
                $x_5$ &  \\
                \hline
                $x_7$ & \ok \\
                \hline
            \end{tabular}
        };
        \node[main] (7) [left of=6] {$x_7$};
        \node[below of=7, rectangle] (desc7) {
            \begin{tabular}{|c|c|}
                \hline
                $x_1$ & \ok \\
                \hline
                $x_2$ &  \\
                \hline
                $x_3$ &  \\
                \hline
                $x_4$ &  \\
                \hline
                $x_5$ &  \\
                \hline
                $x_6$ &  \\
                \hline
            \end{tabular}
            };
            \draw[->] (1) -- (2);
            \draw[->] (2) -- (3);
            \draw[->] (2) -- (4);
            \draw[->] (4) -- (5);
            \draw[->] (5) -- (6);
            \draw[->] (6) -- (7);
            \draw[->] (7) -- (1);
    \end{tikzpicture}

    \vspace{1cm}
    A partir desse momento todos os processos já foram testados pelo menos uma vez. Então,
    todos passam por mais um ciclo passando a informação uma a uma para o processo posterior.
    \vspace{1cm}

    \begin{tikzpicture}[node distance={20mm}, thick, main/.style = {draw, circle}] 
        \node[main] (1) {$x_1$};
        \node[above left of=1, rectangle] (desc1) {
            \begin{tabular}{|c|c|}
                \hline
                $x_2$ & \ok \\
                \hline
                $x_3$ & \fail \\
                \hline
                $x_4$ &  \\
                \hline
                $x_5$ &  \\
                \hline
                $x_6$ &  \\
                \hline
                $x_7$ &  \\
                \hline
            \end{tabular}
        };
        \node[main] (2) [above right of=1] {$x_2$};
        \node[above of=2, rectangle] (desc2) {
            \begin{tabular}{|c|c|}
                \hline
                $x_1$ &  \\
                \hline
                $x_3$ & \fail \\
                \hline
                $x_4$ & \ok \\
                \hline
                $x_5$ &  \\
                \hline
                $x_6$ &  \\
                \hline
                $x_7$ &  \\
                \hline
            \end{tabular}
        };
        \node[main] (3) [right of=2] {$x_3$}; 
        \node[above of=3, rectangle] (desc3) {
            \begin{tabular}{|c|c|}
                \hline
                $x_1$ &  \\
                \hline
                $x_2$ &  \\
                \hline
                $x_4$ &  \\
                \hline
                $x_5$ &  \\
                \hline
                $x_6$ &  \\
                \hline
                $x_7$ &  \\
                \hline
            \end{tabular}
        };
        \node[main] (4) [below right of=3] {$x_4$};
        \node[right of=4, rectangle] (desc4) {
            \begin{tabular}{|c|c|}
                \hline
                $x_1$ &  \\
                \hline
                $x_2$ &  \\
                \hline
                $x_3$ &  \\
                \hline
                $x_5$ & \ok \\
                \hline
                $x_6$ &  \\
                \hline
                $x_7$ &  \\
                \hline
            \end{tabular}
        };
        \node[main] (5) [below left of=4] {$x_5$}; 
        \node[below of=5, rectangle] (desc5) {
            \begin{tabular}{|c|c|}
                \hline
                $x_1$ &  \\
                \hline
                $x_2$ &  \\
                \hline
                $x_3$ &  \\
                \hline
                $x_4$ &  \\
                \hline
                $x_6$ & \ok \\
                \hline
                $x_7$ &  \\
                \hline
            \end{tabular}
        };
        \node[main] (6) [left of=5] {$x_6$};
        \node[below of=6, rectangle] (desc6) {
            \begin{tabular}{|c|c|}
                \hline
                $x_1$ &  \\
                \hline
                $x_2$ &  \\
                \hline
                $x_3$ &  \\
                \hline
                $x_4$ &  \\
                \hline
                $x_5$ &  \\
                \hline
                $x_7$ & \ok \\
                \hline
            \end{tabular}
        };
        \node[main] (7) [left of=6] {$x_7$};
        \node[below of=7, rectangle] (desc7) {
            \begin{tabular}{|c|c|}
                \hline
                $x_1$ & \ok \\
                \hline
                $x_2$ &  \\
                \hline
                $x_3$ &  \\
                \hline
                $x_4$ &  \\
                \hline
                $x_5$ &  \\
                \hline
                $x_6$ &  \\
                \hline
            \end{tabular}
            };
            \draw[->] (1) -- (2);
            \draw[->] (2) -- (3);
            \draw[->] (2) -- (4);
            \draw[->] (4) -- (5);
            \draw[->] (5) -- (6);
            \draw[->] (6) -- (7);
            \draw[->] (7) -- (1);
    \end{tikzpicture}

    \begin{tikzpicture}[node distance={20mm}, thick, main/.style = {draw, circle}] 
        \node[main] (1) {$x_1$};
        \node[above left of=1, rectangle] (desc1) {
            \begin{tabular}{|c|c|}
                \hline
                $x_2$ & \ok \\
                \hline
                $x_3$ & \fail \\
                \hline
                $x_4$ &  \\
                \hline
                $x_5$ &  \\
                \hline
                $x_6$ &  \\
                \hline
                $x_7$ &  \\
                \hline
            \end{tabular}
        };
        \node[main] (2) [above right of=1] {$x_2$};
        \node[above of=2, rectangle] (desc2) {
            \begin{tabular}{|c|c|}
                \hline
                $x_1$ &  \\
                \hline
                $x_3$ & \fail \\
                \hline
                $x_4$ & \ok \\
                \hline
                $x_5$ & \ok \\
                \hline
                $x_6$ &  \\
                \hline
                $x_7$ &  \\
                \hline
            \end{tabular}
        };
        \node[main] (3) [right of=2] {$x_3$}; 
        \node[above of=3, rectangle] (desc3) {
            \begin{tabular}{|c|c|}
                \hline
                $x_1$ &  \\
                \hline
                $x_2$ &  \\
                \hline
                $x_4$ &  \\
                \hline
                $x_5$ &  \\
                \hline
                $x_6$ &  \\
                \hline
                $x_7$ &  \\
                \hline
            \end{tabular}
        };
        \node[main] (4) [below right of=3] {$x_4$};
        \node[right of=4, rectangle] (desc4) {
            \begin{tabular}{|c|c|}
                \hline
                $x_1$ &  \\
                \hline
                $x_2$ &  \\
                \hline
                $x_3$ &  \\
                \hline
                $x_5$ & \ok \\
                \hline
                $x_6$ &  \\
                \hline
                $x_7$ &  \\
                \hline
            \end{tabular}
        };
        \node[main] (5) [below left of=4] {$x_5$}; 
        \node[below of=5, rectangle] (desc5) {
            \begin{tabular}{|c|c|}
                \hline
                $x_1$ &  \\
                \hline
                $x_2$ &  \\
                \hline
                $x_3$ &  \\
                \hline
                $x_4$ &  \\
                \hline
                $x_6$ & \ok \\
                \hline
                $x_7$ &  \\
                \hline
            \end{tabular}
        };
        \node[main] (6) [left of=5] {$x_6$};
        \node[below of=6, rectangle] (desc6) {
            \begin{tabular}{|c|c|}
                \hline
                $x_1$ &  \\
                \hline
                $x_2$ &  \\
                \hline
                $x_3$ &  \\
                \hline
                $x_4$ &  \\
                \hline
                $x_5$ &  \\
                \hline
                $x_7$ & \ok \\
                \hline
            \end{tabular}
        };
        \node[main] (7) [left of=6] {$x_7$};
        \node[below of=7, rectangle] (desc7) {
            \begin{tabular}{|c|c|}
                \hline
                $x_1$ & \ok \\
                \hline
                $x_2$ &  \\
                \hline
                $x_3$ &  \\
                \hline
                $x_4$ &  \\
                \hline
                $x_5$ &  \\
                \hline
                $x_6$ &  \\
                \hline
            \end{tabular}
            };
            \draw[->] (1) -- (2);
            \draw[->] (2) -- (3);
            \draw[->] (2) -- (4);
            \draw[->] (4) -- (5);
            \draw[->] (5) -- (6);
            \draw[->] (6) -- (7);
            \draw[->] (7) -- (1);
    \end{tikzpicture}

    \vspace{1cm}
    Depois de todos os processos passarem seus resultados um a um para os próximos, teremos:
    \vspace{1cm}

    \begin{tikzpicture}[node distance={20mm}, thick, main/.style = {draw, circle}] 
        \node[main] (1) {$x_1$};
        \node[above left of=1, rectangle] (desc1) {
            \begin{tabular}{|c|c|}
                \hline
                $x_2$ & \ok \\
                \hline
                $x_3$ & \fail \\
                \hline
                $x_4$ & \ok \\
                \hline
                $x_5$ & \ok \\
                \hline
                $x_6$ & \ok \\
                \hline
                $x_7$ & \ok \\
                \hline
            \end{tabular}
        };
        \node[main] (2) [above right of=1] {$x_2$};
        \node[above of=2, rectangle] (desc2) {
            \begin{tabular}{|c|c|}
                \hline
                $x_1$ & \ok \\
                \hline
                $x_3$ & \fail \\
                \hline
                $x_4$ & \ok \\
                \hline
                $x_5$ & \ok \\
                \hline
                $x_6$ & \ok \\
                \hline
                $x_7$ & \ok \\
                \hline
            \end{tabular}
        };
        \node[main] (3) [right of=2] {$x_3$}; 
        \node[above of=3, rectangle] (desc3) {
            \begin{tabular}{|c|c|}
                \hline
                $x_1$ &  \\
                \hline
                $x_2$ & \\
                \hline
                $x_4$ &  \\
                \hline
                $x_5$ &  \\
                \hline
                $x_6$ &  \\
                \hline
                $x_7$ &  \\
                \hline
            \end{tabular}
        };
        \node[main] (4) [below right of=3] {$x_4$};
        \node[right of=4, rectangle] (desc4) {
            \begin{tabular}{|c|c|}
                \hline
                $x_1$ & \ok \\
                \hline
                $x_2$ & \ok \\
                \hline
                $x_3$ & \fail \\
                \hline
                $x_5$ & \ok \\
                \hline
                $x_6$ & \ok \\
                \hline
                $x_7$ & \ok \\
                \hline
            \end{tabular}
        };
        \node[main] (5) [below left of=4] {$x_5$}; 
        \node[below of=5, rectangle] (desc5) {
            \begin{tabular}{|c|c|}
                \hline
                $x_1$ & \ok \\
                \hline
                $x_2$ & \ok \\
                \hline
                $x_3$ & \fail \\
                \hline
                $x_4$ & \ok \\
                \hline
                $x_6$ & \ok \\
                \hline
                $x_7$ & \ok \\
                \hline
            \end{tabular}
        };
        \node[main] (6) [left of=5] {$x_6$};
        \node[below of=6, rectangle] (desc6) {
            \begin{tabular}{|c|c|}
                \hline
                $x_1$ & \ok \\
                \hline
                $x_2$ & \ok \\
                \hline
                $x_3$ & \fail \\
                \hline
                $x_4$ & \ok \\
                \hline
                $x_5$ & \ok \\
                \hline
                $x_7$ & \ok \\
                \hline
            \end{tabular}
        };
        \node[main] (7) [left of=6] {$x_7$};
        \node[below of=7, rectangle] (desc7) {
            \begin{tabular}{|c|c|}
                \hline
                $x_1$ & \ok \\
                \hline
                $x_2$ & \ok \\
                \hline
                $x_3$ & \fail \\
                \hline
                $x_4$ & \ok \\
                \hline
                $x_5$ & \ok \\
                \hline
                $x_6$ & \ok \\
                \hline
            \end{tabular}
            };
            \draw[->] (1) -- (2);
            \draw[->] (2) -- (3);
            \draw[->] (2) -- (4);
            \draw[->] (4) -- (5);
            \draw[->] (5) -- (6);
            \draw[->] (6) -- (7);
            \draw[->] (7) -- (1);
    \end{tikzpicture}

\end{document}