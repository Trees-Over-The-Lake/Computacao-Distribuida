\documentclass{article} 
\usepackage{tikz}
\usepackage[portuguese]{babel}

\usetikzlibrary{automata,positioning,shapes.multipart}

\title{Algoritmo de VRing com erro no node 3}
\author{Lucas Santiago de Oliveira}
\date{\today}

\begin{document} 
    \maketitle

    Aqui estão os 7 processos. Nenhum deles tem conhecimento do funcionamento dos outros.
    \vspace{1cm}

    \begin{tikzpicture}[node distance={20mm}, thick, main/.style = {draw, circle}] 
        \node[main] (1) {$x_1$};
        \node[above left of=1, rectangle] (desc1) {
            \begin{tabular}{|c|c|}
                \hline
                $x_2$ &  \\
                \hline
                $x_3$ &  \\
                \hline
                $x_4$ &  \\
                \hline
                $x_5$ &  \\
                \hline
                $x_6$ &  \\
                \hline
                $x_7$ &  \\
                \hline
            \end{tabular}
        };
        \node[main] (2) [above right of=1] {$x_2$};
        \node[above of=2, rectangle] (desc2) {
            \begin{tabular}{|c|c|}
                \hline
                $x_1$ &  \\
                \hline
                $x_3$ &  \\
                \hline
                $x_4$ &  \\
                \hline
                $x_5$ &  \\
                \hline
                $x_6$ &  \\
                \hline
                $x_7$ &  \\
                \hline
            \end{tabular}
        };
        \node[main] (3) [right of=2] {$x_3$}; 
        \node[above of=3, rectangle] (desc3) {
            \begin{tabular}{|c|c|}
                \hline
                $x_1$ &  \\
                \hline
                $x_2$ &  \\
                \hline
                $x_4$ &  \\
                \hline
                $x_5$ &  \\
                \hline
                $x_6$ &  \\
                \hline
                $x_7$ &  \\
                \hline
            \end{tabular}
        };
        \node[main] (4) [below right of=3] {$x_4$};
        \node[right of=4, rectangle] (desc4) {
            \begin{tabular}{|c|c|}
                \hline
                $x_1$ &  \\
                \hline
                $x_2$ &  \\
                \hline
                $x_3$ &  \\
                \hline
                $x_5$ &  \\
                \hline
                $x_6$ &  \\
                \hline
                $x_7$ &  \\
                \hline
            \end{tabular}
        };
        \node[main] (5) [below left of=4] {$x_5$}; 
        \node[below of=5, rectangle] (desc5) {
            \begin{tabular}{|c|c|}
                \hline
                $x_1$ &  \\
                \hline
                $x_2$ &  \\
                \hline
                $x_3$ &  \\
                \hline
                $x_4$ &  \\
                \hline
                $x_6$ &  \\
                \hline
                $x_7$ &  \\
                \hline
            \end{tabular}
        };
        \node[main] (6) [left of=5] {$x_6$};
        \node[below of=6, rectangle] (desc6) {
            \begin{tabular}{|c|c|}
                \hline
                $x_1$ &  \\
                \hline
                $x_2$ &  \\
                \hline
                $x_3$ &  \\
                \hline
                $x_4$ &  \\
                \hline
                $x_5$ &  \\
                \hline
                $x_7$ &  \\
                \hline
            \end{tabular}
        };
        \node[main] (7) [left of=6] {$x_7$};
        \node[below of=7, rectangle] (desc7) {
            \begin{tabular}{|c|c|}
                \hline
                $x_1$ &  \\
                \hline
                $x_2$ &  \\
                \hline
                $x_3$ &  \\
                \hline
                $x_4$ &  \\
                \hline
                $x_5$ &  \\
                \hline
                $x_6$ &  \\
                \hline
            \end{tabular}
        };
    \end{tikzpicture}

    \vspace{1cm}
    Aqui o processo 1 
    \vspace{1cm}

    \begin{tikzpicture}[node distance={20mm}, thick, main/.style = {draw, circle}] 
        \node[main] (1) {$x_1$}; 
        \node[main] (2) [above right of=1] {$x_2$};
        \node[main] (3) [right of=2] {$x_3$}; 
        \node[main] (4) [below right of=3] {$x_4$};
        \node[main] (5) [below left of=4] {$x_5$}; 
        \node[main] (6) [left of=5] {$x_6$};
        \node[main] (7) [left of=6] {$x_7$};  
    \end{tikzpicture} 
\end{document}

% Modelo de grafo 
%
% \begin{tikzpicture}[node distance={20mm}, thick, main/.style = {draw, circle}] 
%     \node[main] (1) {$x_1$};
%     \node[above left of=1, rectangle] (desc1) {
%         \begin{tabular}{|c|c|}
%             \hline
%             $x_2$ &  \\
%             \hline
%             $x_3$ &  \\
%             \hline
%             $x_4$ &  \\
%             \hline
%             $x_5$ &  \\
%             \hline
%             $x_6$ &  \\
%             \hline
%             $x_7$ &  \\
%             \hline
%         \end{tabular}
%     };
%     \node[main] (2) [above right of=1] {$x_2$};
%     \node[above of=2, rectangle] (desc2) {
%         \begin{tabular}{|c|c|}
%             \hline
%             $x_1$ &  \\
%             \hline
%             $x_3$ &  \\
%             \hline
%             $x_4$ &  \\
%             \hline
%             $x_5$ &  \\
%             \hline
%             $x_6$ &  \\
%             \hline
%             $x_7$ &  \\
%             \hline
%         \end{tabular}
%     };
%     \node[main] (3) [right of=2] {$x_3$}; 
%     \node[above of=3, rectangle] (desc3) {
%         \begin{tabular}{|c|c|}
%             \hline
%             $x_1$ &  \\
%             \hline
%             $x_2$ &  \\
%             \hline
%             $x_4$ &  \\
%             \hline
%             $x_5$ &  \\
%             \hline
%             $x_6$ &  \\
%             \hline
%             $x_7$ &  \\
%             \hline
%         \end{tabular}
%     };
%     \node[main] (4) [below right of=3] {$x_4$};
%     \node[right of=4, rectangle] (desc4) {
%         \begin{tabular}{|c|c|}
%             \hline
%             $x_1$ &  \\
%             \hline
%             $x_2$ &  \\
%             \hline
%             $x_3$ &  \\
%             \hline
%             $x_5$ &  \\
%             \hline
%             $x_6$ &  \\
%             \hline
%             $x_7$ &  \\
%             \hline
%         \end{tabular}
%     };
%     \node[main] (5) [below left of=4] {$x_5$}; 
%     \node[below of=5, rectangle] (desc5) {
%         \begin{tabular}{|c|c|}
%             \hline
%             $x_1$ &  \\
%             \hline
%             $x_2$ &  \\
%             \hline
%             $x_3$ &  \\
%             \hline
%             $x_4$ &  \\
%             \hline
%             $x_6$ &  \\
%             \hline
%             $x_7$ &  \\
%             \hline
%         \end{tabular}
%     };
%     \node[main] (6) [left of=5] {$x_6$};
%     \node[below of=6, rectangle] (desc6) {
%         \begin{tabular}{|c|c|}
%             \hline
%             $x_1$ &  \\
%             \hline
%             $x_2$ &  \\
%             \hline
%             $x_3$ &  \\
%             \hline
%             $x_4$ &  \\
%             \hline
%             $x_5$ &  \\
%             \hline
%             $x_7$ &  \\
%             \hline
%         \end{tabular}
%     };
%     \node[main] (7) [left of=6] {$x_7$};
%     \node[below of=7, rectangle] (desc7) {
%         \begin{tabular}{|c|c|}
%             \hline
%             $x_1$ &  \\
%             \hline
%             $x_2$ &  \\
%             \hline
%             $x_3$ &  \\
%             \hline
%             $x_4$ &  \\
%             \hline
%             $x_5$ &  \\
%             \hline
%             $x_6$ &  \\
%             \hline
%         \end{tabular}
%     };
% \end{tikzpicture}