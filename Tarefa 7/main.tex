\documentclass{article}
\usepackage[portuguese]{babel}
\usepackage{indentfirst}


\title{Tarefa 7 - Relógios Lógicos e Físicos}
\author{Lucas Santiago de Oliveira}
\date{\today}

\begin{document}
    \maketitle

    \section{Funcionamento de Relógios Lógicos}

    Relógios lógicos são uma invenção da computação para sincronizar processos dentro de um sistema distribuído.
    Em um sistema distribuído vários processos acontecem em simultaneidade, por conta disso, é necessário que
    exista alguma forma de saber qual processo foi executado primeiro, para manter todo o sistema ordenado. 
    No momento em que um processo novo é criado, um contador lógico também é iniciado para esse processo. 
    Se esse precisar se comunicar com outro processo, o uso desse contador se torna imprecindível para manter
    a comunicação entre processos sincronizada, sem que um processo envie uma mensagem $T_2$ antes da $T_1$,
    ou seja, fora de ordem.

    \section{Algoritmo de Cristian para Relógios}

    O algoritmo de Cristian foi criado para sincronizar computadores a partir de um computador com relógio preciso.
    Outros computadores ao consultarem esse computador preciso sofrem com um grande atraso da latência do envio e do
    recebimento da mensagem na rede. O algoritmo de Cristian foi projetado para resolver esse problema ao somar
    um pequeno valor de tempo à mensagem de envio para os computadores, prevendo que haverá algum atraso na rede.

    Esse algoritmo foi projetado para funcionar em ambientes locais, pois em sistemas distantes pode ocorrer das 
    mensagem sofrerem uma latência maior, devido à alteração de rota ou atraso de um dos roteadores na rede,
    do que a esperada pelo algoritmo, com isso, recebendo um valor incorreto de tempo na máquina local.

    \section{Ordenando os Processos}

    Processo A: inst A1; send C; recv B; inst A2; send D; recv D; inst A3; 

    Processo B: send A; inst B1; recv C; inst B2; recv D; 

    Processo C: inst C1; inst C2; recb A; inst C3; send B; 
    
    Processo D: inst D1; recv A; send A; inst D2; send B

    \subsection*{Ordem da execução dos processos:}

    Início
    
    Execução 1: 
    
    Simultanemente: (Processo A $\rightarrow$ inst A1; Processo B $\rightarrow$ Entra em modo de espera por A;
    Processo C $\rightarrow$ inst C1; Processo D $\rightarrow$ instD1;) 

    Execução 2:

    Simultanemente: (Processo A $\rightarrow$ envia mensagem para C; Processo B $\rightarrow$ ainda em espera de A;
    Processo C $\rightarrow$  inst C2; Processo D $\rightarrow$ recebe mensagem de A)

    Execução 3:

    Simultanemente: (Processo A $\rightarrow$ recebe mensagem de B; Processo B $\rightarrow$  envia a mensagem
    para A; Processo C $\rightarrow$ entra em espera para receber a mensagem para A; Processo D $\rightarrow$ 
    entra em espera para enviar mensagem para A)

    Execução 4:

    Simultanemente: (Processo A $\rightarrow$ inst A2; Processo B $\rightarrow$ Entra em espra do processo C; 
    Processo C $\rightarrow$ inst C3; Processo D $\rightarrow$ inst D2)

    Execução 5:

    Simultanemente: (Processo A $\rightarrow$ envia mensagem para D; Processo B $\rightarrow$ inst B2; 
    Processo C $\rightarrow$ inst C3; Processo D $\rightarrow$ inst D2)

    Fim

\end{document}