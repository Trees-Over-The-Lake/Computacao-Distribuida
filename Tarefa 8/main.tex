\documentclass{article}
\usepackage[portuguese]{babel}
\usepackage{indentfirst}

\title{Tarefa 8}
\author{Lucas Santiago de Oliveira}
\date{7 de outubro de 2022}

\begin{document}
    \maketitle

    \section*{Perguntas}
    \begin{enumerate}
        \item Um determinado sistema distribuído pode não ser justo, permitindo que um processo use uma seção crítica 
        muitas vezes, antes de outro processo. Como os algoritmos de exclusão mútua de Lamport e de Ricart e Agrawala 
        se comportam, cada um, em relação a essa situação? Por que? 
        \item Faça uma resenha do artigo: Maekawa, “A $\sqrt{N}$ Algorithm for Mutual Exclusion in in Decentralized Systems”, 
        ACM Trans. on Computer Systems, Vol. 3, No. 2, 1995..
    \end{enumerate}

    \section*{Minhas Respostas}
    \begin{enumerate}
        \item A forma escolhida por Lamport para solucionar o problema de \emph{fairness} em sistemas distribuídos
        se baseia em organizar todas as requisições em uma fila perfeita. Para que essa fila esteja ordenada, todo
        novo processo que faz uma requisição adiciona seu próprio \emph{timestamp} de criação e os envia para todos 
        os nós. Com essa data de criação adicionada fica fácil de ordenar a fila por ordem de chegada. Com a fila 
        ordenada, sempre que a sessão crítica estiver disponível o próximo processo da fila pode entrar nela.

        Ricart e Agrawala prepararam uma melhoria do algoritmo de Lamport. Não sendo mais necessário um canal perfeito
        de fila. Uma vez que precisa da aprovação de todos os processos para acessar a área crítica. Isso só ocorre 
        no momento em que não há mais nenhum processo em região crítica e, além disso, é o próximo à entrar na região 
        crítica para todos os processos.
        
        \item A ideia de Maekawa era de dividir os computadores na rede em quorums. Com isso, era possível que um quorum 
        tivesse apenas um computador para decidir se o outro quorum poderia entrar na região crítica. Assim, não era necessário
        obter a resposta de todos os nós da rede. Apenas um nó do quorum tinha permissão suficiente para aprovar outro quorum.
        Essa abordagem é $\sqrt{N}$, pois um quorum só mandava mensagem para um computador em todos os outros quorums, não
        necessitando de resposta de todos os nós.
    \end{enumerate}

\end{document}